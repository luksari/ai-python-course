\documentclass[12pt,a4paper]{article}
\usepackage[polish]{babel}
\usepackage{polski}
\usepackage[utf8]{inputenc}
\usepackage[T1]{fontenc}
\usepackage{graphicx}
\usepackage{lmodern}
\usepackage{listings}
\usepackage{color}

\DeclareUnicodeCharacter{2010}{-}
\graphicspath{ {./} }

\definecolor{dkgreen}{rgb}{0,0.6,0}
\definecolor{gray}{rgb}{0.5,0.5,0.5}
\definecolor{mauve}{rgb}{0.58,0,0.82}

\lstset{frame=tb,
  language=python,
  aboveskip=3mm,
  belowskip=3mm,
  showstringspaces=false,
  columns=flexible,
  basicstyle={\small\ttfamily},
  numbers=none,
  numberstyle=\tiny\color{gray},
  keywordstyle=\color{blue},
  commentstyle=\color{dkgreen},
  stringstyle=\color{mauve},
  breaklines=true,
  breakatwhitespace=true,
  tabsize=3
}

\addtolength{\hoffset}{-1.5cm}
\addtolength{\marginparwidth}{-1.5cm}
\addtolength{\textwidth}{3cm}
\addtolength{\voffset}{-1cm}
\addtolength{\textheight}{2.5cm}
\setlength{\topmargin}{0cm}
\setlength{\headheight}{0cm}

\begin{document}
	
	\title{Sprawozdanie nr III\\Systemy Sztucznej Inteligencji}
	\author{Łukasz Tyszkiewicz, grupa I/A}
	\date{\today}
	
	\maketitle
	\begin{itemize}
		\item 1.Wczytać zbiór uczący iris i dokonać jego podziału na część uczącą i testową (po 75 próbek dla uczenia i testowania) 
	\begin{lstlisting}
from sklearn.datasets import load_iris
from sklearn.model_selection import train_test_split

iris = load_iris()
X = iris.data
y = iris.target
 
train, test, train_targets, test_targets = train_test_split(X, y,
 test_size=0.50) 

print("Wielkosc zbioru treningowego:",len(train))
print("Wielkosc zbioru testowego:",len(test))                
	\end{lstlisting}
		\begin{figure}[h]
                        \includegraphics[width=0.8\textwidth]{01}
                        \centering
			\caption{Rozwiązanie zadania 1}
			\label{fig:fig1}
                \end{figure}
                \clearpage 

                \item 2.Skonstruować drzewo klasyfikacyjne dla domyślnych wartości parametrów na podstawie zbioru uczącego i dokonać jego wizualizacji
	\begin{lstlisting}
from sklearn.datasets import load_iris
from sklearn.model_selection import train_test_split
from sklearn import tree
import pydot
import io
import graphviz
                
iris = load_iris()
X = iris.data
y = iris.target
train, test, train_targets, test_targets = train_test_split(X, y,       test_size=0.50)
clf = tree.DecisionTreeClassifier()
clf = clf.fit(X, y)
y = clf.predict(X)
                
dot_data = io.StringIO()
tree.export_graphviz(clf, out_file=dot_data)
graph = pydot.graph_from_dot_data(dot_data.getvalue())
graph[0].write_png("02.png")
                        \end{lstlisting}
		\begin{figure}[h]
                        \includegraphics[width=0.8\textwidth]{02}
                        \centering
			\caption{Rozwiązanie zadania 2}
			\label{fig:fig2}
                \end{figure}
                \clearpage 

                \item 3.Ocenić uzyskaną sprawność klasyfikacji na zbiorze testowym. Ile elementów zostało niepoprawnie zaklasyfikowanych?
	\begin{lstlisting}
                from sklearn.datasets import load_iris
                from sklearn.model_selection import train_test_split
                from sklearn import tree
                import math
                
                iris = load_iris()
                X = iris.data
                y = iris.target
                train, test, train_targets, test_targets = train_test_split(
                    X, y,       test_size=0.50)
                clf = tree.DecisionTreeClassifier()
                clf = clf.fit(X, y)
                
                k = 10
                clfScore = clf.score(train, test_targets)
                print("Sprawnosc klasyfikatora",clfScore * 100, "%")
                print("Niepoprawnie zaklasyfikowanych: ",
                      math.ceil(len(train) * (1 - clfScore)))
                
	\end{lstlisting}
		\begin{figure}[h]
                        \includegraphics[width=0.6\textwidth]{03}
                        \centering
			\caption{Rozwiązanie zadania 3}
			\label{fig:fig3}
                \end{figure}
                \clearpage

                \item 4.Odczytać wartości parametrów drzewa klasyfikacyjnego. Jakie kryterium decyduje o wyborze testu dla wartości atrybutów?
        \begin{lstlisting}
                print("O wyborze testu decyduje indeks Giniego")
	\end{lstlisting}
		\begin{figure}[h]
                        \includegraphics[width=0.8\textwidth]{04}
                        \centering
			\caption{Rozwiązanie zadania 4}
			\label{fig:fig4}
                \end{figure}
                \clearpage

        \item 5.Przetestować działanie algorytmu drzewa klasyfikacyjnego na próbkach zbioru testowego i ocenić sprawność klasyfikacji.
        	\begin{lstlisting}
from sklearn.datasets import load_iris
from sklearn.model_selection import train_test_split
from sklearn import tree


iris = load_iris()
X = iris.data
y = iris.target
train, test, train_targets, test_targets = train_test_split(
    X, y,       test_size=0.50)
clf = tree.DecisionTreeClassifier()
clf = clf.fit(X, y)
y = clf.predict(X)

score = (iris.target == y).sum()
print("Poprawnie zaklasyfikowanych. : ", score)
print("Sprawnosc: ", float(score) / len(y))

	\end{lstlisting}
		\begin{figure}[h]
                        \includegraphics[width=0.6\textwidth]{05}
                        \centering
			\caption{Rozwiązanie zadania 5}
			\label{fig:fig5}
                \end{figure}
                \clearpage

        \item 6.Skonstruować i wyświetlić drzewo ponownie ograniczając jego głębokość do dwóch oraz trzech. Jaką w tym przypadku osiągamy sprawność klasyfikacji?
        \begin{lstlisting}
                
	\end{lstlisting}
		\begin{figure}[h]
                        \includegraphics[width=0.6\textwidth]{061}
                        \centering
			\caption{Rozwiązanie zadania 6}
			\label{fig:fig6}
                \end{figure}
                \begin{figure}[h]
                        \includegraphics[width=0.6\textwidth]{062}
                        \centering
			\caption{Rozwiązanie zadania 6}
			\label{fig:fig6.1}
                \end{figure}
                \clearpage

                        \item 7.Skonstruować drzewo klasyfikacyjne korzystając z kryterium przyrostu informacji dla wyboru testu (Wskazówka: criterion=’entropy’).
	\begin{lstlisting}
clf = NearestCentroid()
clf.fit(train, train_targets)
Z = clf.predict(test)

c1 = (Z == 1).nonzero()
c2 = (Z == 2).nonzero()
plt.scatter(test[c1, 0], test[c1, 1], c="g", label="Klasa 1")
plt.scatter(test[c2, 0], test[c2, 1], c="r", label="Klasa 2")
plt.legend()
plt.scatter(clf.centroids_[:, 0], clf.centroids_[:, 1], c="b")
plt.show()
	\end{lstlisting}
		\begin{figure}[h]
                        \includegraphics[width=0.6\textwidth]{07}
                        \centering
			\caption{Rozwiązanie zadania 7}
			\label{fig:fig7}
                \end{figure}
                \clearpage

                \item 8.Przetestować uzyskane drzewo na zbiorze testowym i porównać wynik z drzewem uzyskanym dla indeksu Giniego.
	\begin{lstlisting}
from sklearn.datasets import load_iris
from sklearn.model_selection import train_test_split
from sklearn import tree
import pydot
import io
import graphviz


iris = load_iris()
X = iris.data
y = iris.target
train, test, train_targets, test_targets = train_test_split(
    X, y,       test_size=0.50)
clf = tree.DecisionTreeClassifier(criterion='entropy')
clf = clf.fit(X, y)
y = clf.predict(X)

score = (iris.target == y).sum()
print("Poprawnie zaklasyfikowanych. : ", score)
print("Sprawnosc: ", float(score) / len(y))

	\end{lstlisting}
		\begin{figure}[h]
                        \includegraphics[width=0.6\textwidth]{08}
                        \centering
			\caption{Rozwiązanie zadania 8}
			\label{fig:fig8}
                \end{figure}
                \clearpage
                \item 9. Skonstruować drzewo klasyfikacyjne dla zbioru iris w podprzestrzeni cech złożonej jedynie z dwóch pierwszych atrybutów. Ocenić sprawność uzyskanego rozwiązania.
	\begin{lstlisting}
from sklearn.datasets import load_iris
from sklearn.model_selection import train_test_split
from sklearn import tree
import pydot
import io
import graphviz


iris = load_iris()
X = iris.data[:, :2]
y = iris.target 
train, test, train_targets, test_targets = train_test_split(X, y,
                                               test_size=0.50)
clf = tree.DecisionTreeClassifier(criterion='entropy',  max_depth=2,
                                  min_samples_split=2, min_samples_leaf=2)
clf = clf.fit(X, y)
y = clf.predict(X)

score = (iris.target == y).sum()
print("Poprawnie zaklasyfikowanych. : ", score)
print("Sprawnosc: ", float(score) / len(y))

	\end{lstlisting}
		\begin{figure}[h]
                        \includegraphics[width=0.6\textwidth]{09}
                        \centering
			\caption{Rozwiązanie zadania 9}
			\label{fig:fig9}
                \end{figure}
                \clearpage
        \end{itemize}
	
\end{document}
\end{document}
