\documentclass[12pt,a4paper]{article}
\usepackage[polish]{babel}
\usepackage[T1]{fontenc}
\usepackage[utf8x]{inputenc}
\usepackage{graphicx}
\usepackage{listings}
\usepackage{color}

\graphicspath{ {./} }

\definecolor{dkgreen}{rgb}{0,0.6,0}
\definecolor{gray}{rgb}{0.5,0.5,0.5}
\definecolor{mauve}{rgb}{0.58,0,0.82}

\lstset{frame=tb,
  language=python,
  aboveskip=3mm,
  belowskip=3mm,
  showstringspaces=false,
  columns=flexible,
  basicstyle={\small\ttfamily},
  numbers=none,
  numberstyle=\tiny\color{gray},
  keywordstyle=\color{blue},
  commentstyle=\color{dkgreen},
  stringstyle=\color{mauve},
  breaklines=true,
  breakatwhitespace=true,
  tabsize=3
}

\addtolength{\hoffset}{-1.5cm}
\addtolength{\marginparwidth}{-1.5cm}
\addtolength{\textwidth}{3cm}
\addtolength{\voffset}{-1cm}
\addtolength{\textheight}{2.5cm}
\setlength{\topmargin}{0cm}
\setlength{\headheight}{0cm}

\begin{document}
	
	\title{Sprawozdanie nr I\\Systemy Sztucznej Inteligencji}
	\author{Łukasz Tyszkiewicz, grupa I/A}
	\date{\today}
	
	\maketitle
	\begin{itemize}
		\item 1.Wczytać zbiór danych iris. Podać liczbę próbek w tym zbiorze oraz ilość atrybutów opisujących każdą z nich.
	\begin{lstlisting}
iris = open("iris.data", 'r')

irisData = []

for line in iris:
        row = line.split(",")
        irisData.append(row)

print("Number of probes: " + " " + str(len(irisData)))

for probe in irisData:
        print("Number of attributes: " + str(len(probe)))

	\end{lstlisting}
		\begin{figure}[h]
                        \includegraphics[width=0.8\textwidth]{first}
                        \centering
			\caption{Rozwiązanie zadania 1}
			\label{fig:fig1}
                \end{figure}
                \clearpage 

                \item 2.Odczytać wartości atrybutów dla próbek o numerach 10 i 75. Obliczyć ich odległość euklidesową.
	\begin{lstlisting}
iris = open("iris.data", 'r')

irisData = []

for line in iris:
        row = line.split(",")
        irisData.append(row)

sum = 0

for i in range(3):
        dif = float(irisData[75][i]) - float(irisData[10][i])
        powDif = m.pow(dif, 2)
        sum = sum + powDif

sqrtSum = m.sqrt(sum)

print("Odleglosc euklidesowa: " + str(sqrtSum))

	\end{lstlisting}
		\begin{figure}[h]
                        \includegraphics[width=0.8\textwidth]{second}
                        \centering
			\caption{Rozwiązanie zadania 2}
			\label{fig:fig2}
                \end{figure}
                \clearpage 

                \item 3.Podać wartości minimalne, maksymalne, średnie i odchylenia standardowe dla każdego z atrybutów
	\begin{lstlisting}
pet_len, pet_width, sep_len, sep_width = [], [], [], []

for probe in irisData:
        pet_len.append(float(probe[2]))
        pet_width.append(float(probe[3]))
        sep_len.append(float(probe[0]))
        sep_width.append(float(probe[1]))


def showData(val, key):
    maxVal = max(val)
    minVal = min(val)
    avgVal = sum(val) / float(len(val))
    stdDevVal = np.std(val, axis=0)
    print("\n")
    print("Maximum " + key + " value: " + str(maxVal))
    print("Minimum " + key + " value: " + str(minVal))
    print("Average " + key + " value: " + str(avgVal)[:7])
    print("Standart deviation of " + key + " value: " + str(stdDevVal)[:7])

showData(pet_len, "petal length")
showData(pet_width, "petal width")
showData(sep_len, "sepal length")
showData(sep_width, "sepal width")
	\end{lstlisting}
		\begin{figure}[h]
                        \includegraphics[width=0.6\textwidth]{third}
                        \centering
			\caption{Rozwiązanie zadania 3}
			\label{fig:fig3}
                \end{figure}
                \clearpage

                \item 4.Dokonać wizualizacji zbioru iris w przestrzeni złożonej z dwóch pierwszych atrybutów.
	\begin{lstlisting}
sep_len = []
sep_width = []
irisData = irisData[:-1]

for probe in irisData:
        sep_len.append(float(probe[0]))
        sep_width.append(float(probe[1]))

plt.plot(sep_len, sep_width, 'ro')
plt.ylabel("Sepal width")
plt.xlabel("Sepal length")
plt.show()
	\end{lstlisting}
		\begin{figure}[h]
                        \includegraphics[width=0.8\textwidth]{fourth}
                        \centering
			\caption{Rozwiązanie zadania 4}
			\label{fig:fig4}
                \end{figure}
                \clearpage

        \item 5.Dokonać wizualizacji zbioru iris w przestrzeni złożonej z atrybutów 1 oraz 3 przy czym elementy każdej z klas zaznaczyć innym kolorem
	\begin{lstlisting}
pet_len, sep_len = [], []
p_lenSet, p_lenVer, p_lenVir, s_lenSet, s_lenVer, s_lenVir = [], [], [], [], [], []

for probe in irisData:
    pet_len.append(float(probe[0]))
    sep_len.append(float(probe[2]))
    if probe[4] == "Iris-setosa":
        p_lenSet.append(float(probe[2]))
        s_lenSet.append(float(probe[0]))
    elif probe[4] == "Iris-versicolor":
        p_lenVer.append(float(probe[2]))
        s_lenVer.append(float(probe[0]))
    elif probe[4] == "Iris-virginica":
        s_lenVir.append(float(probe[0]))
        p_lenVir.append(float(probe[2]))
max_sep_len = max(sep_len)
max_pet_len = max(pet_len)
min_pet_len = min(pet_len)
min_sep_len = min(sep_len)
offset = 0.25

plt.plot(p_lenSet, s_lenSet, 'ro')
plt.plot(p_lenVer, s_lenVer, 'go')
plt.plot(p_lenVir, s_lenVir, 'bo')
plt.ylabel("Petal length")
plt.xlabel("Sepal length")
plt.show()
	\end{lstlisting}
		\begin{figure}[h]
                        \includegraphics[width=0.6\textwidth]{fifth}
                        \centering
			\caption{Rozwiązanie zadania 5}
			\label{fig:fig5}
                \end{figure}
                \clearpage

        \item 6.Podać średnią wartość zmierzonych atrybutów dla próbek z klasy setosa oraz versicolor.
	\begin{lstlisting}
_class, sep_len, sep_width, pet_len, pet_width = [], [], [], [], []

for probe in irisData:
    _class.append(probe[4])
    sep_len.append(float(probe[0]))
    sep_width.append(float(probe[1]))
    pet_len.append(float(probe[2]))
    pet_width.append(float(probe[3]))

data = np.column_stack((_class, sep_len, sep_width, pet_len, pet_width))

avg_pet_len = []
avg_pet_width = []
avg_sep_len = []
avg_sep_width = []


def showData(key):

    pet_len = []
    sep_len = []
    pet_width = []
    sep_width = []

    index = 0
    for val in data:
        if val[0] == key:
            sep_len.append(float(val[1]))
            sep_width.append(float(val[2]))
            pet_len.append(float(val[3]))
            pet_width.append(float(val[4]))

    avg_pet_len.append(sum(pet_len) / len(pet_len))
    avg_pet_width.append(sum(pet_width) / len(pet_width))
    avg_sep_width.append(sum(sep_width) / len(sep_width))
    avg_sep_len.append(sum(sep_len) / len(sep_len))

    if key == "Iris-setosa":
        index = 0
    elif key == "Iris-versicolor":
        index = 1
    elif key == "Iris-virginica":
        index = 2
    print("\n" + key + " average attributes values:"
          + "\npetal length: " + str(avg_pet_len[index])[:5]
          + "\npetal width: " + str(avg_pet_width[index])[:5]
          + "\nsepal length: " + str(avg_sep_len[index])[:5]
          + "\nsepal width: " + str(avg_sep_width[index])[:5])


showData("Iris-setosa")
showData("Iris-versicolor")
showData("Iris-virginica")
	\end{lstlisting}
		\begin{figure}[h]
                        \includegraphics[width=0.6\textwidth]{sixth}
                        \centering
			\caption{Rozwiązanie zadania 6}
			\label{fig:fig6}
                \end{figure}
                \clearpage

                        \item 7.Dane poddać normalizacji i po dokonaniu tej operacji obliczyć ponownie wartości minimalne, maksymalne, średnie i odchylenia standardowe dla każdego z atrybutów.
	\begin{lstlisting}
dataMatrix = np.matrix((pet_len, pet_width, sep_len, sep_width))
dataNormalized = preprocessing.normalize(dataMatrix, norm="l1")

pet_len = dataNormalized[0]
pet_width = dataNormalized[1]
sep_len = dataNormalized[2]
sep_width = dataNormalized[3]


def showData(val, key):
    maxVal = max(val)
    minVal = min(val)
    avgVal = sum(val) / float(len(val))
    stdDevVal = np.std(val, axis=0)
    print("\n")
    print("Maximum " + key + " value: " + str(maxVal))
    print("Minimum " + key + " value: " + str(minVal))
    print("Average " + key + " value: " + str(avgVal)[:7])
    print("Standart deviation of " + key + " value: " + str(stdDevVal)[:7])


showData(pet_len, "petal length")
showData(pet_width, "petal width")
showData(sep_len, "sepal length")
showData(sep_width, "sepal width")
	\end{lstlisting}
		\begin{figure}[h]
                        \includegraphics[width=0.6\textwidth]{seventh}
                        \centering
			\caption{Rozwiązanie zadania 7}
			\label{fig:fig7}
                \end{figure}
                \clearpage

                                       \item 8.Wygenerować losowo zbiór 10 danych w przestrzeni dwuwymiarowej. Pierwszy atrybut z rozkładu N(−2,1)N(−2,1), drugi z rozkładu jednostajnego na przedziale [0,10][0,10]. Zbiór danych zwizualizować za pomocą wykresu.
	\begin{lstlisting}
first_attr = np.random.rand(10) - 2
second_attr = np.random.uniform(0, 10, 10)
data = np.column_stack((first_attr, second_attr))

plt.plot(first_attr, second_attr, 'ro')
plt.xlabel("N(-2,1) Distribiution")
plt.ylabel("[0,10] Distribiution")
plt.show()

	\end{lstlisting}
		\begin{figure}[h]
                        \includegraphics[width=0.6\textwidth]{eight}
                        \centering
			\caption{Rozwiązanie zadania 8}
			\label{fig:fig8}
                \end{figure}
                \clearpage
                                       \item 9.Podać macierz odległości euklidesowych, mahalanobisa oraz Minkowskiego L1L1 dla wszystkich par elementów tego zbioru
	\begin{lstlisting}
first_attr = np.random.randn(10)-2
second_attr = 10*np.random.randn(10)
data = np.vstack((first_attr, second_attr))
data=data.conj().transpose()

euclidean_matrix = sklearn.metrics.pairwise.pairwise_distances(data, metric='euclidean')
mahalanobian_matrix = sklearn.metrics.pairwise.pairwise_distances(data,  metric='mahalanobis')
minkowskian_matrix = sklearn.metrics.pairwise.pairwise_distances(data, metric='minkowski')

	\end{lstlisting}
		\begin{figure}[h]
                        \includegraphics[width=0.6\textwidth]{nineth}
                        \centering
			\caption{Rozwiązanie zadania 9}
			\label{fig:fig9}
                \end{figure}
                \clearpage
                                       \item Dokonać skalowania liniowego wygenerowanego zbioru na przedział [0,1][0,1] i ponownie obliczyć odległości dla wszystkich par obiektów.
	\begin{lstlisting}
data = np.vstack((first_attr, second_attr))
data=data.conj().transpose()

normalize = preprocessing.MinMaxScaler((0,1))
data = normalize.fit_transform(data)

euclidean_matrix = sklearn.metrics.pairwise.pairwise_distances(
    data, metric='euclidean')
mahalanobian_matrix = sklearn.metrics.pairwise.pairwise_distances(
    data,  metric='mahalanobis')
minkowskian_matrix = sklearn.metrics.pairwise.pairwise_distances(
    data, metric='minkowski')

	\end{lstlisting}
		\begin{figure}[h]
                        \includegraphics[width=0.6\textwidth]{tenth}
                        \centering
			\caption{Rozwiązanie zadania 10}
			\label{fig:fig10}
                \end{figure}
                \clearpage
\item 11.Zaproponować postać funkcji klasyfikujących dla problemu klasyfikacyjnego dla dwóch klas i dwuwymiarowej przestrzeni cech przy założeniu, że elementy klasy 1 znajdują się w drugiej, a elementy klasy 2 w czwartej ćwiartce układu współrzędnych. Podać wzór określający powierzchnię decyzyjną tego klasyfikatora.
	\begin{lstlisting}
def classifier(x):
    if x[0] >= 0 and x[1] >= 0:
        return 1
    elif x[0] < 0 and x[1] > 0:
        return 2
    elif x[0] <= 0 and x[1] <= 0:
        return 3
    elif x[0] > 0 and x[1] <0:
        return 4

labels=np.array([classifier(data[i,:]) for i in range(2*N)])

c1 = (labels==1).nonzero()
c2 = (labels==2).nonzero()
c3 = (labels==3).nonzero()
c4 = (labels==4).nonzero()

	\end{lstlisting}
		\begin{figure}[h]
                        \includegraphics[width=0.6\textwidth]{eleven}
                        \centering
			\caption{Rozwiązanie zadania 11}
			\label{fig:fig11}
                \end{figure}
                \clearpage
\item 12.Wygenerować przykładowy zbiór danych testowych (po 10 próbek na klasę) dla problemu z poprzedniego zadania. Dokonać testowania zaproponowanego klasyfikatora.
	\begin{lstlisting}
class1 = np.random.normal(1, 8, 2*N)
class2 = 10*np.random.rand(1,2*N) -7

data = np.vstack((class1, class2))
data = data.conj().transpose()



min_x, min_y, max_x, max_y, offset = min(data[:, 0]), min(
    data[:, 1]), max(data[:, 0]), max(data[:, 1]), 1

fig = plt.figure()
ax = fig.add_subplot(111)

plt.scatter(data[:, 0], data[:, 1])

left, right = ax.get_xlim()
low, high = ax.get_ylim()


plt.plot([-20, 20], [0, 0], "black")
plt.plot([0, 0], [-20, 20], "black")

plt.grid()

plt.show()

labels = np.array([classifier(data[i, :]) for i in range(2*N)])

c1 = (labels == 1).nonzero()
c2 = (labels == 2).nonzero()
c3 = (labels == 3).nonzero()
c4 = (labels == 4).nonzero()

fig = plt.figure()
ax = fig.add_subplot(111)

plt.scatter(data[:, 0], data[:, 1])

left, right = ax.get_xlim()
low, high = ax.get_ylim()

plt.plot([-20, 20], [0, 0], "black")
plt.plot([0, 0], [-20, 20], "black")
plt.grid()

plt.scatter(data[c1, 0], data[c1, 1], c='b')
plt.scatter(data[c2, 0], data[c2, 1], c='r')
plt.scatter(data[c3, 0], data[c3, 1], c='g')
plt.scatter(data[c4, 0], data[c4, 1], c='y')

	\end{lstlisting}
		\begin{figure}[h]
                        \includegraphics[width=0.6\textwidth]{twelve}
                        \centering
			\caption{Rozwiązanie zadania 12}
			\label{fig:fig12}
                \end{figure}
                \clearpage

        	\end{itemize}
	
\end{document}
